\documentclass[fontsize=12pt,paper=a4,pagesize,headings=small]{scrartcl}
\usepackage{lmodern}
\usepackage[T1]{fontenc}
\usepackage[utf8]{inputenc}
\usepackage[ngerman]{babel}
\renewcommand*{\othersectionlevelsformat}[3]{\S\,#3\autodot\enskip}
\renewcommand\thesection{§ \arabic{section}}

\title{Satzung des Geany e.\,V.}
\author{Geany e. V.}
\date{In der Fassung vom 19. November 2022}

\begin{document}
\maketitle{}

\section{Name, Sitz und Geschäftsjahr}
\begin{enumerate}
    \item Der Name des Vereins lautet »Geany«. Der Verein soll in das
    Vereinsregister beim Amtsgericht Stendal eingetragen und der Name
    dann um den Zusatz »e.V.« ergänzt werden.

    \item Der Verein hat seinen Sitz in Halle (Saale). Sofern keine
    feste Geschäftsstelle eingerichtet ist, folgt die Verwaltung dem
    Wohnort des jeweiligen Vorstandsmitglieds, das die Geschäftsführung
    wahrnimmt.

    \item Geschäftsjahr ist das Kalenderjahr.
\end{enumerate}

\section{Vereinszweck}
\begin{enumerate}
    \item Der Zweck des Vereins ist die Förderung, Verbreitung und
    Weiterentwicklung Freier Software, insbesondere der
    Entwicklungsumgebung Geany, unter den Grundsätzen des freien
    Wissensaustauschs. Gleichzeitiges Ziel des Vereins ist die
    Chancengleichheit beim Zugang zur Software sowie Volksbildung im
    Bereich des Umgangs mit Geany, Freier Software und deren Lizenzen.
    Darüber hinaus ist es Ziel des Vereins, durch sein Tun aktiv
    Wissenschaft und Forschung zu unterstützen.

    \item Der Satzungszweck wird insbesondere verwirklicht durch

    %inspired by KDE e.V.%
    \begin{enumerate}
        \item die Förderung der Bildung, des Meinungsaustauschs und der
        Zusammenarbeit von Anwendern, Entwicklern und Forschern

        \item die Weiterentwicklung und Forschung an freier Software,
        insbesondere Geany

        \item die Bereitstellung von Dokumentation sowie
        Förderung der Verfügbarkeit und die Erstellung und
        Verbreitung von Informationsmaterial

        \item Beiträge zur sachkundigen Information der Öffentlichkeit
        im Tätigkeitsbereich des Vereins und Teilnahme an Messen und
        Kongressen um die Informationen einem breiteren Spektrum von
        Anwendern zugänglich zu machen

        \item Organisation von Kongressen und allgemein zugänglichen
        Vorträgen zur Weiterbildung der Projektteilnehmer und Anwender

        \item das Bewahren der freien Rechte der Projektteilnehmer zum
        Schutz vor kommerziellen Interessen Dritter.
    \end{enumerate}
\end{enumerate}

\section{Gemeinnützigkeit}
\begin{enumerate}

    \item Der Verein verfolgt im Rahmen seiner Tätigkeit gemäß § 2 der
    Satzung ausschließlich und unmittelbar gemeinnützige Zwecke im
    Sinne des Abschnittes steuerbegünstigte Zwecke der Abgabenordnung
    (§§ 51ff.\ AO). Er ist selbstlos tätig und verfolgt nicht in erster
    Linie eigenwirtschaftliche Zwecke.

    \item Die Mittel des Vereins sind ausschließlich zu satzungsgemäßen
    Zwecken zu verwenden. Die Mitglieder erhalten ausschließlich
    Erstattungen entstandener Kosten, aber keine direkten Zuwendungen
    aus Mitteln des Verein.

    \item Niemand darf durch Vereinsausgaben, die dem Vereinszweck
    fremd sind oder durch unverhältnismäßig hohe Vergütungen begünstigt
    werden.
\end{enumerate}

\section{Mitgliedschaft}
\begin{enumerate}
    \item Mitglieder des Vereins können natürliche und juristische
    Personen werden, die die Ziele des Vereins mittragen und
    unterstützen wollen.

    \item Über die Aufnahme von Mitgliedern entscheidet der Vorstand.

    \item Die Beitrittserklärung erfolgt schriftlich gemäß
    \ref{Textform} gegenüber dem Vorstand. Die Mitgliedschaft
    beginnt mit der Aushändigung einer entsprechenden Bestätigung durch
    ein Vorstandsmitglied.

    \item Hat der Vorstand die Aufnahme abgelehnt, so kann der
    Mitgliedschaftsbewerber Einspruch zur nächsten
    Mitgliederversammlung einlegen, die daraufhin abschließend über die
    Aufnahme oder Nichtaufnahme entscheidet.

    \item Die Mitgliedschaft endet durch Austrittserklärung, durch
    Ausschluss, durch Tod einer natürlichen Personen, oder durch
    Auflösung und Erlöschung von juristischen Personen. Die
    Beitragspflicht für das laufende Geschäftsjahr wird von der
    Geschäftsordnung geregelt.

    \item Der Austritt wird durch eine gemäß \ref{Textform}
    schriftliche Willenserklärung gegenüber dem Vorstand erklärt.

    \item Das Instrument des Vereinsausschlusses ist kritischen
    Situationen vorbehalten, wobei grundsätzlich der Klärung zur Güte
    der Vorrang zu gewähren ist. Der Ausschluss erfolgt auf Beschluss
    des Vorstandes mit sofortiger Wirkung. Gründe für einen Ausschluss
    können sein:
        \begin{enumerate}
            \item ein schwerer Verstoß eines Mitglieds gegen die in
                dieser Satzung festgelegten Bestimmungen sowie Ziele und
                Zwecke des Vereins nach einem erfolglosen Versuch der
                Klärung, sowie
            \item ein trotz mehrfacher Mahnung bestehender Rückstand an
                Beitragszahlungen über einen Zeitraum von 12 Monaten.
        \end{enumerate}

    Das Mitglied muss über den Ausschluss informiert werden und ihm vor der
    Beschlussfassung Gelegenheit zur Rechtfertigung bzw.\ Stellungnahme
    gegeben werden. Gegen den Ausschluss kann innerhalb von vier Wochen
    beim Vorstand Widerspruch eingelegt werden, über den die nächste
    Mitgliederversammlung entscheidet. Bis zur Entscheidung der
    Mitgliederversammlung ruhen die Rechte und Pflichten des Mitglieds.

    \item Bei Ausscheiden eines Mitglieds aus dem Verein oder bei
    Vereinsauflösung besteht kein Anspruch auf Rückerstattung etwa
    eingebrachter Vermögenswerte.
\end{enumerate}

\section{Rechte und Pflichten der Mitglieder}

\begin{enumerate}
    \item Ordentliche Mitglieder sind berechtigt, die Leistungen des
        Vereins entsprechend der vorhandenen Möglichkeiten und in angemessenem und verhältnismäßigem Ausmaß in Anspruch zu nehmen.

    \item Mitglieder sind verpflichtet, die satzungsgemäßen Zwecke des
        Vereins zu unterstützen und zu fördern.

    \item Der Verein erhebt einen Mitgliedsbeitrag, zu dessen Zahlung die
        Mitglieder verpflichtet sind. Näheres regelt eine Geschäftsordnung,
        die von der Mitgliederversammlung beschlossen wird.
\end{enumerate}

\section{Organe des Vereins}
\begin{enumerate}
    \item Die Organe des Vereins sind:
        \begin{itemize}
            \item Die Mitgliederversammlung;
            \item Der Vorstand.
        \end{itemize}
\end{enumerate}

\section{Mitgliederversammlung}

\begin{enumerate}
    \item Die Mitgliederversammlung ist das oberste Beschlussorgan des
    Vereins. Ihr obliegen alle Entscheidungen, die nicht durch die
    Satzung oder die Geschäftsordnung einem anderen Organ übertragen
    wurden.

    \item Beschlüsse werden von der Mitgliederversammlung durch
    öffentliche Abstimmung getroffen. Auf Wunsch eines ordentlichen
    Mitglieds ist geheim abzustimmen.

    \item Jedes ordentliche Mitglied hat genau eine Stimme.

    \item Juristische Personen können zu der Mitgliederversammlung
    einen Vertreter senden, der an Diskussionen und Abstimmungen im
    Auftrag der juristischen Person teilnimmt. Eine Wahl des Vertreters
    für Ämter des Vereins wird ausgeschlossen.

    \item Zur Fassung eines Beschlusses ist in der Regel eine relative
    Mehrheit der abgegebenen Stimmen notwendig. Ausgenommen sind die in
    \ref{satzungsaenderung} und \ref{aufloesung} geregelten
    Angelegenheiten. Auf Antrag und aus gegebenem Anlass kann eine
    Abstimmung auch durch das Zustimmwahlrecht erfolgen. Während dieser
    hat jedes ordentliches Mitglied mehrere Stimmen.

    \item Eine ordentliche Mitgliederversammlung, bezeichnet als
    Jahreshauptversammlung, wird einmal jährlich einberufen. Ihre
    Tagesordnung umfasst unter anderem den Rechenschaftsbericht des
    Vorstandes über die Vereinstätigkeit sowie den Rechenschaftsbericht
    des Schatzmeisters für das vorherige Geschäftsjahr.

    \item Die Mitgliederversammlung erfolgt entweder real oder virtuell
    (Onlineverfahren) in einem nur für Mitglieder mit ihren
    Legitimationsdaten und einem gesonderten Zugangswort zugänglichen
    Chat-Raum.
    Im Onlineverfahren wird das jeweils nur für die aktuelle Versammlung
    gültige Zugangswort mit einer gesonderten Email unmittelbar vor der
    Versammlung, maximal 3 Stunden davor, bekannt gegeben.

    \item Nimmt mindestens ein ordentliches Mitglied über eine elektronische
    Möglichkeit an der Mitgliederversammlung teil, sind Entschlüsse der
    Versammlung auf vorherigen Antrag durch Briefwahl oder durch
    vergleichbare sichere elektronische Wahlformen zu fassen.

    \item Eine außerordentliche Mitgliederversammlung kann jederzeit
    einberufen werden, wenn mindestens 23\% der ordentlichen Mitglieder
    oder der Vorstand dies jeweils schriftlich gemäß \ref{Textform}
    unter Angabe eines Grunds beantragen. Dem angegebenen Grund müssen
    die gewünschten Tagesordnungspunkte zu entnehmen sein; sie werden
    auf die Einladung übernommen.
    Die Mitgliederversammlung ist als Präsenzversammlung durchzuführen,
    soweit dies mit dem Verlangen beantragt wird.

    \item Dem Vorstand obliegt zu allen Mitgliederversammlungen die
    Festsetzung eines Termins, des Ortes und die rechtzeitige Einladung
    aller Mitglieder bis spätestens zwei Wochen vor dem von ihm
    festgesetzten Termin. Bei von den Mitgliedern beantragten
    Mitgliederversammlungen darf der Termin nicht mehr als acht Wochen
    nach dem Eingang des Antrags beim Vorstand liegen.

    \item Der Vorstand kann die Einladungen auf schriftlichem Weg gemäß
    \ref{Textform} zustellen, muss jedoch eine Kopie auf dem
    Postweg zustellen, falls das Mitglied den Wunsch dazu schriftlich
    gemäß \ref{Textform} angemeldet hat.

    \item In der Einladung werden die Tagesordnungspunkte sowie weitere
    nötige Informationen bekannt gegeben. Die Mitgliederversammlung
    kann per Beschluss die Tagesordnung verändern, sofern keine
    gesetzliche Regelung etwas anderes bestimmt. Insbesondere bei
    Tagesordnungspunkten zur Änderungen der Satzung, der
    Geschäftsordnung oder der Wahl von Vorstandsmitgliedern ist eine
    Änderung der Tagesordnung ausgeschlossen.

    \item Über die Beschlüsse der Mitgliederversammlung ist ein
    Protokoll anzufertigen, das vom Versammlungsleiter oder einem
    anwesenden Vorstandsmitglied zu unterzeichnen ist. Das Protokoll
    ist innerhalb von 14 Tagen allen Mitgliedern zugänglich zu machen
    und auf der nächsten Mitgliederversammlung genehmigen zu lassen.

    \item Der Vorstandsvorsitzende ist Versammlungsleiter der
    Mitgliederversammlung. Die Mitgliederversammlung kann durch
    Beschluss einen anderen Versammlungsleiter oder Schriftführer
    bestimmen.
\end{enumerate}

\section{Vorstand}\label{vorstand}
\begin{enumerate}
    \item Der Vorstand besteht aus mindestens drei ordentlichen
    Mitgliedern: dem Vorstandsvorsitzenden, dem Schatzmeister und dem
    Schriftführer. Des Weiteren können bis zu drei Beisitzer in den
    Vorstand gewählt werden. Es kann auf Wunsch der
    Mitgliederversammlung auf eine Wahl der Beisitzer verzichtet
    werden.

    \item Vorstand im Sinne des § 26 BGB sind der Vorstandsvorsitzender,
    Schatzmeister sowie der Schriftführer. Diese sind einzeln
    berechtigt, den Verein nach außen zu vertreten. Die
    Geschäftsordnung kann hierfür Einschränkungen festlegen.

    \item Vorstandsmitglieder können jederzeit von ihrem Amt
    zurücktreten.

    \item Bei Rücktritt oder andauernder Ausübungsunfähigkeit eines
    Vorstandsmitglieds ist der gesamte Vorstand neu zu wählen. Bis zur
    Wahl eines neuen Vorstandes ist der bisherige Vorstand zur
    bestmöglichen Wahrnehmung seiner Aufgaben verpflichtet.

    \item Die Amtsdauer der Vorstandsmitglieder beträgt zwei Jahre. Sie
    werden von der Mitgliederversammlung aus den ordentlichen
    Mitgliedern des Vereins gewählt. Es werden nacheinander
    Vorstandsvorsitzender, Schatzmeister und Schriftführer sowie falls
    gewünscht bis zu drei Beisitzer gewählt. Eine Wiederwahl in dieser
    Funktion ist beliebig oft zulässig. Vertreter von juristischen
    können nicht für ein solches Amt kandidieren.

    \item Der Vorstand ist Dienstvorgesetzter aller vom Verein
    angestellten Mitarbeiter. Er kann diese Aufgabe an einen
    Dritten übertragen.

    \item Die Vorstandsmitglieder sind grundsätzlich ehrenamtlich
    tätig. Sie haben Anspruch auf Erstattung notwendiger Auslagen,
    deren Rahmen von der Geschäftsordnung festgelegt wird.

    \item Der Vorstand tritt nach Bedarf zusammen. Die
    Vorstandssitzungen werden vom Schriftführer schriftlich gemäß
    \ref{Textform} einberufen. Der Vorstand ist beschlussfähig,
    wenn mindestens zwei Drittel der Vorstandsmitglieder anwesend sind.
    Die Beschlüsse der Vorstandssitzung sind schriftlich zu
    protokollieren.

    \item Jedes Vorstandsmitglied hat bei Abstimmungen des Vorstands
    eine Stimme. Bei Abstimmungen ist eine Mehrheit von zwei Dritteln
    der abgegebenen gültigen Stimmen nötig.
\end{enumerate}

\section{Satzungs- und Geschäftsordnungsänderung}\label{satzungsaenderung}
\begin{enumerate}
    \item Über Satzungs- und Geschäftsordnungsänderungen kann in der
        Mitgliederversamm\-lung nur abgestimmt werden, wenn auf diesen
        Tagesordnungspunkt hingewiesen wurde und der Einladung sowohl der
        bisherige als auch der vorgesehene neue Text beigefügt worden war.

    \item Für die Satzungs- oder Geschäftsordnungsänderung ist eine
        Mehrheit von zwei Dritteln in der Mitgliederversammlung erforderlich.

    \item Satzungsänderungen, die von Aufsichts-, Gerichts- oder
        Finanzbehörden aus formalen Gründen verlangt werden, kann der
        Vorstand von sich aus vornehmen. Diese Satzungsänderungen müssen der
        nächsten Mitgliederversammlung mitgeteilt werden.
\end{enumerate}


\section{Auflösung des Vereins und Vermögensbindung}\label{aufloesung}
\begin{enumerate}

    \item Die Auflösung des Vereins muss von der Mitgliederversammlung mit
        einer Mehrheit von drei Vierteln beschlossen werden. Die Abstimmung
        ist nur möglich, wenn auf der Einladung zur Mitgliederversammlung
        als einziger Tagesordnungspunkt die Auflösung des Vereins
        angekündigt wurde.

    \item Bei Auflösung des Vereins oder Aufhebung der Körperschaft
        darf das Vermögen der Körperschaft nur für steuerbegünstigte Zwecke
        verwendet werden. Zur Erfüllung dieser Voraussetzung wird das
        Vermögen einer anderen steuerbegünstigten Körperschaft oder einer
        Körperschaft öffentlichen Rechts für steuerbegünstigte Zwecke
        übertragen, die ebenfalls den Auftrag zur Bildung und Volksbildung
        im Umgang mit Informationstechnologie wahrnimmt. Näheres kann die
        Geschäftsordnung regeln.

    \item Der Grundsatz der Vermögensbindung ist bei der Fassung von
        Beschlüssen über die künftige Verwendung des Vereinsvermögens
        zwingend zu erfüllen.
\end{enumerate}

\section{Textform}\label{Textform}
\begin{enumerate}
    \item Schriftliche Erklärungen im Sinne dieser Satzung können auch
        elektronische Dokumente sein. Die Geschäftsordnung bestimmt
        Anforderungen, Zustellwege und Zuordnung derartiger Dokumente.
\end{enumerate}

\newpage
~
\end{document}
